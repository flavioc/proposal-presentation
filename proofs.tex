
\section{Proof Theory}

\begin{frame}[fragile]
   \frametitle{Proof Theory}
   \begin{itemize}
      \item LM is based on a fragment of linear logic extended with definitions (for comprehensions and aggregates)
      \item Designed two dynamic semantics:
      \begin{itemize}
         \item High Level Dynamic (HLD) Semantics:
         \begin{itemize}
            \item Sequent calculus + focusing with positive atoms
            \item To apply a rule, \emph{focus} on the rule as an implication
            \item Rule selection and comprehension application are non-determinism and do not fully describe the behavior of LM
         \end{itemize}
         \item Low Level Dynamic (LLD) Semantics
         \begin{itemize}
            \item Close to the real implementation
            \item Describes rule selection and rule matching using a continuation stack
            \item Comprehensions are maximally applied
         \end{itemize}
      \end{itemize}
      \item The semantics model a single rule application:
      \begin{itemize}
         \item Input: Linear context ($\Delta$), Persistent Context ($\Gamma$), Rules ($\Phi$)
         \item Output: Derived Linear context ($\Delta'$), Derived Persistent context ($\Gamma'$), Consumed Linear context ($\Xi'$)
      \end{itemize}
   \end{itemize}
\end{frame}

\begin{frame}[fragile]
   \frametitle{Proof Theory: HLD}
   \begin{block}{Application}
\[
\infer[\az rule]
{\az \Gamma ; \Delta_1, \Delta_2 ; A \lolli B \rightarrow \Xi' ; \Delta' ; \Gamma'}
{\mz \Gamma ; \Delta_1 \rightarrow A & \dz \Gamma ; \Delta_2; \Delta_1; \cdot ; \cdot ; B \rightarrow \Xi' ; \Delta' ; \Gamma'}
\]

\[
\infer[\doz rule]
{\doz \Gamma ; \Delta ; R, \Phi \rightarrow \Xi' ; \Delta' ; \Gamma'}
{\az \Gamma ; \Delta ; R \rightarrow \Xi' ; \Delta' ; \Gamma'}
\]
   \end{block}
\end{frame}

\begin{frame}[fragile]
   \frametitle{Proof Theory: HLD}
   \begin{block}{Match}
\[
\infer[\mz 1]
{\mz \Gamma; \cdot \rightarrow 1}
{}
\tab
\infer[\mz p]
{\mz \Gamma; p \rightarrow p }
{}
\]

\[
\infer[\mz \bang p]
{\mz \Gamma, p; \cdot \rightarrow \bang p}
{}
\tab
\infer[\mz \otimes]
{\mz \Gamma; \Delta_1, \Delta_2 \rightarrow A \otimes B}
{\mz \Gamma; \Delta_1 \rightarrow A & \mz \Delta_2 \rightarrow B}
\]
   \end{block}
\end{frame}

\begin{frame}[fragile]
   \frametitle{Proof Theory: HLD}
   \begin{block}{Derivation}
{\small
      \[
\infer[\dz p]
{\dz \Gamma ; \Delta ; \Xi ; \Gamma_1 ; \Delta_1 ; p, \Omega \rightarrow \Xi' ; \Delta' ; \Gamma'}
{\dz \Gamma ; \Delta ; \Xi ; \Gamma_1 ; p, \Delta_1 ; \Omega \rightarrow \Xi' ; \Delta' ; \Gamma'}
\]

\[
\infer[\dz \bang p]
{\dz \Gamma ; \Delta ; \Xi ; \Gamma_1 ; \Delta_1 ; \bang p, \Omega \rightarrow \Xi' ; \Delta' ; \Gamma'}
{\dz \Gamma ; \Delta ; \Xi ; \Gamma_1, p ; \Delta_1 ; \Omega \rightarrow \Xi' ; \Delta' ; \Gamma'}
\]

\[
\infer[\dz \otimes]
{\dz \Gamma ; \Delta ; \Xi ; \Gamma_1 ; \Delta_1 ; A \otimes B, \Omega \rightarrow \Xi' ; \Delta' ; \Gamma'}
{\dz \Gamma ; \Delta ; \Xi ; \Gamma_1 ; \Delta_1 ; A, B, \Omega \rightarrow \Xi' ; \Delta' ; \Gamma'}
\]

\[
\infer[\dz 1]
{\dz \Gamma ; \Delta ; \Xi ; \Gamma_1; \Delta_1 ; 1, \Omega \rightarrow \Xi' ; \Delta' ; \Gamma'}
{\dz \Gamma ; \Delta ; \Xi ; \Gamma_1; \Delta_1 ; \Omega \rightarrow \Xi' ; \Delta' ; \Gamma'}
\]

\[
\infer[\dz end]
{\dz \Gamma ; \Delta ; \Xi' ; \Gamma' ; \Delta' ; \cdot \rightarrow \Xi' ; \Delta' ; \Gamma'}
{}
\]
}
   \end{block}
\end{frame}

\begin{frame}[fragile]
   \frametitle{Proof Theory: HLD}
   \begin{block}{Derivation}
{\small
      \[
      \infer[\dz comp]
      {\dz \Gamma ; \Delta ; \Xi ; \Gamma_1 ; \Delta_1 ; \comp A \lolli B, \Omega \rightarrow \Xi' ; \Delta' ; \Gamma'}
      {\dz \Gamma ; \Delta ; \Xi ; \Gamma_1 ; \Delta_1 ; 1 \with (A \lolli B \otimes \comp A \lolli B), \Omega \rightarrow \Xi' ; \Delta' ; \Gamma'}
      \]

      \[
      \infer[\dz \lolli]
      {\dz \Gamma ; \Delta_a, \Delta_b ; \Xi ; \Gamma_1 ; \Delta_1 ; A \lolli B, \Omega \rightarrow \Xi' ; \Delta' ; \Gamma'}
      {\mz \Gamma ; \Delta_a \rightarrow A & \dz \Gamma ; \Delta_b ; \Xi, \Delta_a ; \Gamma_1 ; \Delta_1 ; B, \Omega \rightarrow \Xi' ; \Delta' ; \Gamma'}
      \]

      \[
      \infer[\dz \with L]
      {\dz \Gamma ; \Delta ; \Xi ; \Gamma_1 ; \Delta_1 ; A \with B, \Omega \rightarrow \Xi' ; \Delta'; \Gamma'}
      {\dz \Gamma ; \Delta ; \Xi ; \Gamma_1 ; \Delta_1 ; A, \Omega \rightarrow \Xi' ; \Delta'; \Gamma'}
      \]

      \[
      \infer[\dz \with R]
      {\dz \Gamma ; \Delta ; \Xi ; \Gamma_1 ; \Delta_1 ; A \with B, \Omega \rightarrow \Xi' ; \Delta' ; \Gamma'}
      {\dz \Gamma ; \Delta ; \Xi ; \Gamma_1 ; \Delta_1 ; B, \Omega \rightarrow \Xi' ; \Delta' ; \Gamma'}
      \]
      }
   \end{block}
\end{frame}

\begin{frame}[fragile]
   \frametitle{Proof Theory: LLD}
   \begin{block}{LLD Semantics Example}
      \includegraphics<1>[height=6.5cm]{lld1.pdf}
      \includegraphics<2>[height=6.5cm]{lld2.pdf}
      \includegraphics<3>[height=6.5cm]{lld3.pdf}
      \includegraphics<4>[height=6.5cm]{lld4.pdf}
      \includegraphics<5>[height=6.5cm]{lld5.pdf}
      \includegraphics<6>[height=6.5cm]{lld6.pdf}
      \includegraphics<7>[height=6.5cm]{lld7.pdf}
      \includegraphics<8>[height=6.5cm]{lld8.pdf}
      \includegraphics<9>[height=6.5cm]{lld9.pdf}
      \includegraphics<10>[height=6.5cm]{lld10.pdf}
   \end{block}
\end{frame}

\begin{frame}[fragile]
   \frametitle{Proof Theory: Soundness Proof}
   \begin{itemize}
      \item Every proof/execution performed in LLD can also be done in HLD (soundness)
      \item Contents of each continuation frame are important:
      \begin{itemize}
         \item Remaining terms to match
         \item Previous matched terms
         \item Available facts and consumed facts
         \item Remaining facts to try
      \end{itemize}
      \pause
      \item Critical points of the proof:
      \begin{itemize}
         \item Maintaining the well-formedness of the continuation stack at all times
         \item Stack must be related to a term $A$ (term to prove) and to the initial database
         \item Comprehensions and databases work by re-using the stack as many times as many backtracks are possible (database iteration!)
      \end{itemize}
      \pause
      \item On the other hand, completeness cannot be proven!
      \pause
      \item Connection between the dynamic semantics of LM and linear logic is established
      \begin{itemize}
         \item Implementation is close enough to LLD semantics
      \end{itemize}
   \end{itemize}
\end{frame}